\documentclass[
	14pt,
	a4paper
]{article}
\usepackage[utf8]{inputenc}
\usepackage[spanish]{babel}
\usepackage{graphicx}
\usepackage{authoraftertitle}

\def\doctype{MASTER TEST PLAN}
\title{Evaluador de microcontroladores para misiones espaciales}
\author{Gonzalo Nahuel Vaca}

\begin{document}

\makeatletter
\begin{titlepage}
	\begin{center}
		\vspace*{1cm}
		
		\Huge
		\textbf{\doctype}
		
		\vspace{0.5cm}
		\LARGE
		\@title
		
		\vspace{1.5cm}
		
		\textbf{\@author}

		\vspace{3.5cm}

		\includegraphics[width=0.8\textwidth]{img/logoFIUBA.pdf}
		
		\vfill
		Facultad de Ingeniería\\
		Universidad de Buenos Aires\\
		Argentina\\
		\today
	\end{center}
\end{titlepage}
\makeatother
\newpage

\section{Introducción}
\label{sec:introduccion}

El objetivo de este documento es detallar todos los aspectos referidos al \emph{Master Test Plan} (Plan Maestro de Pruebas) del proyecto ``\MyTitle ''.
Su función será determinar si un integrado de calificación comercial puede ser utilizado en una misión espacial. Además se espera que permita evaluar distintas técnicas de mitigación de errores.
El proyecto en desarrollo consiste de un firmware de auto comprobación y un sistema de inyección de \emph{soft-errors}.
Finalmente, se proponen los siguientes subsistemas:

\begin{itemize}
	\item Firmware de auto comprobación:
		\begin{itemize}
			\item Validación de \emph{CANBUS}.
			\item Validación de \emph{SPI}.
			\item Validación de \emph{Watchdog}.
			\item Validación de \emph{UART}.
			\item Generador de informe de secuencia.
		\end{itemize}
	\item Sistema de inyección de \emph{soft-errors}:
		\begin{itemize}
			\item Consola de usuario.
			\item Controlador de ensayos.
			\item Interfaz OCD.
			\item Interfaz serie.
			\item Persistencia de datos.
			\item Generador de informes.
		\end{itemize}
\end{itemize}


\section{Asignaciones}
\label{sec:asignaciones}


\subsection{Responsable}
\label{sub:responsable}

El responsable de la elaboración de este documento es el ingeniero a cargo del desarrollo del proyecto, \MyAuthor.

\subsection{Contratista}
\label{sub:contratista}

La asignación es ejecutada bajo responsabilidad de \MyAuthor, jefe de \emph{testing} del desarrollo de este proyecto.

\subsection{Alcances}
\label{sub:alcances}

El alcance del test de aceptación es el ``\MyTitle '', versión 1.0.

\subsection{Objetivos}
\label{sub:objetivos}

Los objetivos son:

\begin{itemize}
	\item Determinar si el sistema cumple con los requerimientos.
	\item Reportar las diferencias entre lo observado y el comportamiento deseado.
\end{itemize}

\end{document}
