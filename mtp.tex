\documentclass[
    11pt,
    spanish,
	a4paper
]{article}
\usepackage[utf8]{inputenc}
\usepackage[spanish]{babel}
\usepackage{graphicx}
\usepackage{authoraftertitle}
\usepackage{float}

\def\doctype{MASTER TEST PLAN}
\title{Evaluador de microcontroladores para misiones espaciales}
\author{Gonzalo Nahuel Vaca}

\begin{document}

\makeatletter
\begin{titlepage}
	\begin{center}
		\vspace*{1cm}
		
		\Huge
		\textbf{\doctype}
		
		\vspace{0.5cm}
		\LARGE
		\@title
		
		\vspace{1.5cm}
		
		\textbf{\@author}

		\vspace{3.5cm}

		\includegraphics[width=0.8\textwidth]{img/logoFIUBA.pdf}
		
		\vfill
		Facultad de Ingeniería\\
		Universidad de Buenos Aires\\
		Argentina\\
		\today
	\end{center}
\end{titlepage}
\makeatother
\newpage

\section{Introducción}
\label{sec:introduccion}

El objetivo de este documento es detallar todos los aspectos referidos al \emph{Master Test Plan} (Plan Maestro de Pruebas) del proyecto ``\MyTitle ''.
Su función será determinar si un integrado de calificación comercial puede ser utilizado en una misión espacial. Además se espera que permita evaluar distintas técnicas de mitigación de errores.
El proyecto en desarrollo consiste de un firmware de auto comprobación y un sistema de inyección de \emph{soft-errors}.
Finalmente, se proponen los siguientes subsistemas:

\begin{itemize}
	\item Firmware de auto comprobación:
		\begin{itemize}
			\item Validación de \emph{CANBUS}.
			\item Validación de \emph{SPI}.
			\item Validación de \emph{Watchdog}.
			\item Validación de \emph{UART}.
			\item Generador de informe de secuencia.
		\end{itemize}
	\item Sistema de inyección de \emph{soft-errors}:
		\begin{itemize}
			\item Consola de usuario.
			\item Controlador de ensayos.
			\item Interfaz OCD.
			\item Interfaz serie.
			\item Persistencia de datos.
			\item Generador de informes.
		\end{itemize}
\end{itemize}


\section{Asignaciones}
\label{sec:asignaciones}


\subsection{Responsable}
\label{sub:responsable}

El responsable de la elaboración de este documento es el ingeniero a cargo del desarrollo del proyecto, \MyAuthor.

\subsection{Contratista}
\label{sub:contratista}

La asignación es ejecutada bajo responsabilidad de \MyAuthor, jefe de \emph{testing} del desarrollo de este proyecto.

\subsection{Alcances}
\label{sub:alcances}

El alcance del test de aceptación es el ``\MyTitle '', versión 1.0.

\subsection{Objetivos}
\label{sub:objetivos}

Los objetivos son:

\begin{itemize}
	\item Determinar si el sistema cumple con los requerimientos.
	\item Reportar las diferencias entre lo observado y el comportamiento deseado.
\end{itemize}


\section{Estrategia general del \emph{test}}
\label{sec:estrategia}

\subsection{Características de calidad}
\label{sub:caracteristicasCalidad}

Se seleccionan solo aquellas características de calidad que tienen un impacto significativo en el producto.

\begin{itemize}
    \item \underline{30\% Conectividad:} El producto necesita una conexión constante con el servidor \emph{OCD} y con el \emph{CoreSight} para poder cumplir su función. La pérdida de conectividad implica que el ensayo realizado no tiene validez.
    \item \underline{30\% Funcionalidad:} Se asigna un alto nivel de importancia ya que las funciones del producto impactarán sobre las desiciones que el cliente tome sobre el diseño de satélites.
    \item \underline{30\% Fiabilidad:} El peso específico asignado a esta característica se justifica por la gran velocidad de las secuencias. Es poco probable que se pueda detectar un error del producto ya que su función es generar \emph{soft-errors}.
    \item \underline{10\% Usabilidad:} El sistema será utilizado para realizar ensayos específicos. La interfaz de usuario debe ser consitente para evitar fallas por errores humanos.
\end{itemize}

\subsection{Asignación de niveles de prueba a las características de calidad}
\label{sub:asignacion}

\begin{table}[H]
	\centering
	\begin{tabular}{r|cccc}
        & Conectividad & Funcionalidad & Fiabilidad & Usabilidad \\ \hline
        IR (\%)     & 30 & 30 & 30 & 10 \\
        Unitaria    &    & ++ &    &    \\
        Integración & ++ &    & ++ &    \\
        Sistema     &    & +  &    & +  \\
        Aceptación  &    & ++ &    & +  \\
        Campo       & +  &    & +  & ++ \\
	\end{tabular}
\end{table}

Se indica a continuación las razones de la asignación de los niveles de prueba:

\begin{itemize}
    \item \underline{Conectividad:}
    \item \underline{Funcionalidad:}
    \item \underline{Fiabilidad:}
    \item \underline{Usabilidad:}
\end{itemize}
\end{document}
