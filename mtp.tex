\documentclass[
    11pt,
    spanish,
	a4paper
]{article}
\usepackage[utf8]{inputenc}
\usepackage[spanish]{babel}
\usepackage{graphicx}
\usepackage{authoraftertitle}
\usepackage{float}
\usepackage{caption}
\captionsetup[table]{labelformat=empty}

\def\doctype{MASTER TEST PLAN}
\title{Evaluador de microcontroladores para misiones espaciales}
\author{Gonzalo Nahuel Vaca}

\begin{document}

\makeatletter
\begin{titlepage}
	\begin{center}
		\vspace*{1cm}
		
		\Huge
		\textbf{\doctype}
		
		\vspace{0.5cm}
		\LARGE
		\@title
		
		\vspace{1.5cm}
		
		\textbf{\@author}

		\vspace{3.5cm}

		\includegraphics[width=0.8\textwidth]{img/logoFIUBA.pdf}
		
		\vfill
		Facultad de Ingeniería\\
		Universidad de Buenos Aires\\
		Argentina\\
		\today
	\end{center}
\end{titlepage}
\makeatother
\newpage

\section{Introducción}
\label{sec:introduccion}

El objetivo de este documento es detallar todos los aspectos referidos al \emph{Master Test Plan} (Plan Maestro de Pruebas) del proyecto ``\MyTitle ''.
Su función será determinar si un integrado de calificación comercial puede ser utilizado en una misión espacial. Además se espera que permita evaluar distintas técnicas de mitigación de errores.
El proyecto en desarrollo consiste de un firmware de auto comprobación y un sistema de inyección de \emph{soft-errors}.
Finalmente, se proponen los siguientes subsistemas:

\begin{itemize}
	\item Firmware de auto comprobación:
		\begin{itemize}
			\item Validación de \emph{CANBUS}.
			\item Validación de \emph{SPI}.
			\item Validación de \emph{Watchdog}.
			\item Validación de \emph{UART}.
			\item Generador de informe de secuencia.
		\end{itemize}
	\item Sistema de inyección de \emph{soft-errors}:
		\begin{itemize}
			\item Consola de usuario.
			\item Controlador de ensayos.
			\item Interfaz OCD.
			\item Interfaz serie.
			\item Persistencia de datos.
			\item Generador de informes.
		\end{itemize}
\end{itemize}


\section{Asignaciones}
\label{sec:asignaciones}


\subsection{Responsable}
\label{sub:responsable}

El responsable de la elaboración de este documento es el ingeniero a cargo del desarrollo del proyecto, \MyAuthor.

\subsection{Contratista}
\label{sub:contratista}

La asignación es ejecutada bajo responsabilidad de \MyAuthor, jefe de \emph{testing} del desarrollo de este proyecto.

\subsection{Alcances}
\label{sub:alcances}

El alcance del test de aceptación es el ``\MyTitle '', versión 1.0.

\subsection{Objetivos}
\label{sub:objetivos}

Los objetivos son:

\begin{itemize}
	\item Determinar si el sistema cumple con los requerimientos.
	\item Reportar las diferencias entre lo observado y el comportamiento deseado.
\end{itemize}


\section{Estrategia general del \emph{test}}
\label{sec:estrategia}

\subsection{Características de calidad}
\label{sub:caracteristicasCalidad}

Se seleccionan solo aquellas características de calidad que tienen un impacto significativo en el producto.

\begin{itemize}
    \item \underline{30\% Conectividad:} El producto necesita una conexión constante con el servidor \emph{On-Chip Debugger (OCD)} y con el \emph{CoreSight} para poder cumplir su función. La pérdida de conectividad implica que el ensayo realizado no tiene validez.
    \item \underline{30\% Funcionalidad:} Se asigna un alto nivel de importancia ya que las funciones del producto impactarán sobre las desiciones que el cliente tome sobre el diseño de satélites.
    \item \underline{30\% Fiabilidad:} El peso específico asignado a esta característica se justifica por la necesidad de mantener el correcto funcionamiento durante todo el ensayo. Una funcionalidad poco confiable generaría una imposibilidad para evaluar al microcontrolador bajo prueba.
    \item \underline{10\% Usabilidad:} El sistema será utilizado para realizar ensayos específicos. La interfaz de usuario debe ser consitente para evitar fallas por errores humanos.
\end{itemize}

\subsection{Asignación de niveles de prueba a las características de calidad}
\label{sub:asignacion}

\begin{table}[H]
	\centering
	\begin{tabular}{r|cccc}
        & Conectividad & Funcionalidad & Fiabilidad & Usabilidad \\ \hline
        IR (\%)     & 30 & 30 & 30 & 10 \\
        Unitaria    &    & ++ &    &    \\
        Integración & ++ &    & ++ &    \\
        Sistema     &    & +  &    & +  \\
        Aceptación  &    & ++ &    & +  \\
        Campo       & +  &    & +  & ++ \\
	\end{tabular}
\end{table}

Se indica a continuación las razones de la asignación de los niveles de prueba:

\begin{itemize}
    \item \underline{Conectividad:} Las pruebas de integración son las pruebas que determinarán el cumplimiento de los requerimientos de conectividad.
        Adicionalmente, las pruebas finales de campo demostrarán la correcta comunicación entre los módulos.
    \item \underline{Funcionalidad:} Las pruebas unitarias sirven para verificar la funcionalidad de cada sub sistema.
        Finalmente, las pruebas de sistema y aceptación confirmarán que el conjunto funciona de la manera esperada.
    \item \underline{Fiabilidad:} El sistema está compuesto por módulos desacoplados.
        Esto significa que la fiabilidad del sistema depende de como se encuentren acoplados.
        Por estas razones, las pruebas de integración tienen un lugar principal y finalmente, se complementan con las pruebas de campo.
    \item \underline{Usabilidad:} Las pruebas principales son las de campo, en particular, experimentar la experiencia de usuario al crear y correr un ensayo.
        Adicionalmente, las pruebas de sistema y aceptación aportan información complementaria.
\end{itemize}

\section{Estrategia por nivel de prueba}
\label{sec:estrategia}

Por cada nivel indicado en \ref{sub:asignacion}, se evalúa la estrategia seleccionada.

\subsection{Selección de características de calidad y determinación de la importancia relativa por nivel de prueba}
\label{sub:importanciaRelativa}

Se indican a continuación, para cada nivel de prueba, las características de calidad y la importancia relativa de cada una de ellas:

\begin{table}[H]
\centering
\caption{Pruebas unitarias}
\begin{tabular}{c|c}
    Característica de calidad & Importancia relativa \\ \hline
    Funcionalidad & 100 \\
\end{tabular}
\end{table}


\begin{table}[H]
\centering
\caption{Pruebas de integración}
\begin{tabular}{c|c}
    Característica de calidad & Importancia relativa \\ \hline
    Conectividad & 50 \\
    Fiabilidad   & 50 \\
\end{tabular}
\end{table}

\begin{table}[H]
\centering
\caption{Pruebas de sistema}
\begin{tabular}{c|c}
    Característica de calidad & Importancia relativa \\ \hline
    Funcionalidad & 50 \\
    Usabilidad    & 50 \\
\end{tabular}
\end{table}

\begin{table}[H]
\centering
\caption{Pruebas de aceptación}
\begin{tabular}{c|c}
    Característica de calidad & Importancia relativa \\ \hline
    Funcionalidad & 75 \\
    Usabilidad    & 25 \\
\end{tabular}
\end{table}

\begin{table}[H]
\centering
\caption{Pruebas de campo}
\begin{tabular}{c|c}
    Característica de calidad & Importancia relativa \\ \hline
    Conectividad & 25 \\
    Fiabilidad   & 25 \\
    Usabilidad   & 50 \\
\end{tabular}
\end{table}

\subsection{División del sistema en subsistemas}
\label{sub:division}

Como se definieron los módulos en la sección \ref{sec:introduccion}, se divide el proyecto en los siguientes subsistemas:

\begin{itemize}
    \item Parte A. Validación de periféricos.
    \item Parte B. Generador de informe de secuencia.
    \item Parte C. Consola de usuario.
    \item Parte D. Controlador de ensayos.
    \item Parte E. Interfaz OCD.
    \item Parte F. Interfaz serie.
    \item Parte G. Persistencia de datos.
    \item Parte H. Generador de informes.
\end{itemize}

\subsection{Determinación de la importancia relativa de los subsistemas}
\label{sub:determinacion}

\begin{table}[H]
\centering
\begin{tabular}{l|c}
    Subsistema & Importancia Relativa (\%) \\ \hline
    Parte A. Validación de periféricos         & 25 \\
    Parte B. Generador de informe de secuencia & 25 \\
    Parte C. Consola de usuario                &  5 \\
    Parte D. Controlador de ensayos            & 10 \\
    Parte E. Interfaz OCD                      & 10 \\
    Parte F. Interfaz serie                    & 10 \\
    Parte G. Persistencia de datos             &  5 \\
    Parte H. Generador de informes             & 10 \\
\end{tabular}
\end{table}

\subsection{Determinación de la importancia de test por combinaciones de subsistema / característica de calidad}
\label{sub:combinacion}

\begin{table}[H]
\centering
\begin{tabular}{r|ccccccccc}
    & IR (\%) & A & B & C & D & E & F & G & H \\ \hline
    Conectividad  & 30 &    & +  &    & ++ & ++ &    &    &    \\
    Funcionalidad & 30 & ++ &    &    & ++ &    &    &    &    \\
    Fiabilidad    & 30 &    & +  &    &    & +  & +  & +  &    \\
    Usabilidad    & 10 &    &    & ++ &    &    &    & +  & ++ \\
\end{tabular}
\end{table}


\subsection{Determinación de las técnicas de test a ser utilizadas}
\label{sub:tecnicas}

\begin{table}[H]
\centering
\begin{tabular}{r|cccccccc}
    & A & B & C & D & E & F & G & H\\ \hline
    State transition testing   & x &   &   & x &   & x &   &   \\
    Control flow test          &   &   & x &   & x &   & x &   \\
    Elementary comparison test &   &   &   &   &   &   & x &   \\
    Classification-tree method & x & x &   & x &   & x &   &   \\
    Evolutionary algorithms    &   &   &   & x &   &   &   &   \\
    Statistical usage testing  &   &   &   &   &   &   &   & x \\
    Rare event testing         & x &   &   & x &   &   &   &   \\
    Mutation analysis          &   &   &   &   &   &   &   &   \\

\end{tabular}
\end{table}

\end{document}
